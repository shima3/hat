\documentclass[twocolumn]{jarticle}
\usepackage{newtxtext}
\usepackage{amsmath,amssymb}
\usepackage{mathtools}
\mathtoolsset{showonlyrefs=true}
% \usepackage{autobreak}

\setlength{\textwidth}{24zw}
%余白の設定ここから
\setlength{\topmargin}{25mm}
\addtolength{\topmargin}{-1in}
\setlength{\oddsidemargin}{15mm}
\addtolength{\oddsidemargin}{-1.1in}
\setlength{\evensidemargin}{20mm}
\addtolength{\evensidemargin}{-1in}
\setlength{\textwidth}{182mm}
\setlength{\textheight}{265mm}
\setlength{\headsep}{0mm}
\setlength{\headheight}{0mm}
\setlength{\topskip}{0mm}
%余白の設定ここまで
%twocolumn(2列)の間の空白の大きさ設定
\columnsep=8mm
\setlength{\baselineskip}{16pt}
\setlength{\headsep}{-5mm}
\renewcommand{\baselinestretch}{0.85}
% 図と図の間のスペース
\setlength\floatsep{0truemm}
% 本文と図の間のスペース
\setlength\textfloatsep{0truemm}
% 本文中の図のスペース
\setlength\intextsep{0pt}
% 図とキャプションの間のスペース
\setlength\abovecaptionskip{0truemm}
\setlength{\leftmargini}{10pt} 
% \setlength{\mathindent}{\parindent}

% ハット記号
\def\hat{\mbox{\^{ }}}
% 縦棒
\def\|{$|$}
% 縦棒の後,改行を許す
% \def\|{ $|$ \allowbreak}
% ・・・
\def\c...{\mbox{$\cdots\mbox{}$}\allowbreak}
% 非終端記号
\newcommand{\NT}[1]{\ensuremath{\langle\mbox{#1}\rangle}\allowbreak}
% 二重鍵括弧
\newcommand{\dlrBrack}[1]{\ensuremath{[\![#1]\allowbreak\!]}}
% 左矢印
\newcommand{\lto}{\ensuremath{\leftarrow}}
% 右矢印
\newcommand{\lrto}{ \ensuremath{\rightarrow} \allowbreak}
% \newcommand{\lrto}{ \ensuremath{\longrightarrow} \allowbreak}
% コード
\newcommand{\C}[1]{\mbox{\tt#1}}
% ドットの前後に空白
\newcommand{\D}{\mbox{\tt\ .\ }}
% Schemeの意味関数
\newcommand{\sS}[1]{\ensuremath{\mathcal{E}}\dlrBrack{\mbox{\tt#1}}}
% \newcommand{\sS}[1]{\ensuremath{\varepsilon}\dlrBrack{\mbox{\tt#1}}}
% Hatの意味関数
\newcommand{\sH}[1]{\ensuremath{\mathcal{H}}\dlrBrack{\mbox{\tt#1}}}
% 継続付き関数
\newcommand{\FC}[2]{\mbox{\tt F.C #1\D#2}}
% 式に含まれる自由変数の集合
\newcommand{\FV}[1]{\ensuremath{\mathcal{F}}\dlrBrack{\C{#1}}}

% ぶら下げ
\newenvironment{hang}[1][\parindent]
  {\def\item{\par\hangindent=#1\noindent}}
  {\par}

\begin{document}
\title{Hat言語の仕様}
\author{島 和之}
\maketitle

HatはSchemeと同様に静的スコープを持つ動的型付けのプログラミング言語である.
ただし,その評価戦略はSchemeとは異なり,名前呼びである.
つまり,引数として与えられた式の評価が必要な場合,呼び出された関数で明示的に評価する必要がある.
一方,使わない引数を評価する無駄を省くことができる.
また,プログラマが遅延評価を用いた独自の制御構造を実現できる.

\section{形式的構文}

この節では,Hatの形式的構文を示す.
記述を簡潔にするため,BNFを以下のように拡張する.
\begin{itemize}
\item\NT{thing}$^*$は\NT{thing}の0個以上の出現を意味する.
\item\NT{thing}$^+$は\NT{thing}の1個以上の出現を意味する.
\end{itemize}
上記のように拡張したBNFで,Hatプログラムの構文\NT{\tt program}の生成規則を以下に示す.
\begin{hang}\tt % \item 019IOlq
\item\NT{program}\lrto
  \NT{includer}$^*$ \NT{definition}$^*$
\item\NT{includer}\lrto
  (include \NT{string})
\item\NT{definition}\lrto
  (defun \NT{symbol} \NT{hat function})
\item\NT{hat function}\lrto
  \hat\ \NT{head} \NT{body}
\item\NT{head}\lrto
  (\NT{symbol}$^*$)
\item\NT{body}\lrto
  \NT{expression}$^+$\par
  \| \NT{expression}$^+$ \NT{hat function}
\item\NT{expression}\lrto\par
  \NT{symbol} \| \NT{datum} \| (\NT{hat function})
\end{hang}

上記で未定義の記号 \NT{variable}, \NT{literal}, \NT{lambda expression} の定義はR5RSと同じである.
\NT{variable}は変数を示す.
\NT{literal}は真偽値,数値,文字列,quoteを付けたシンボルやリストなどを示す.
\NT{lambda expression}はラムダ式を示す.

\section{形式的意味論}

この節ではHatプログラムに対する形式的な表示的意味論を定める.
以下のように記号を定義する.

\[\begin{array}{rcll}
I      &\in& \mbox{Ide}           & \mbox{識別子(変数)}\\
E      &\in& \mbox{Exp}           & \mbox{式} \\
\rho   &\in& U = \mbox{Ide} \to L & 環境  \\
\kappa &\in& K = E* \to C         & 式の継続 \\
\end{array}\]

\begin{eqnarray*}
v, v_1, v_2,\ldots &\in& \NT{hat variable} \\
f, f_1, f_2,\ldots &\in& \NT{hat function} \\
c, c_1, c_2,\ldots &\in& \NT{hat call} \\
e, e_1, e_2,\ldots &\in& \NT{hat expression} \\
L, L_1, L_2,\ldots &\in& \NT{lambda expression}
\end{eqnarray*}

\C{(defineCPS $v$ $f$)}は$(v, f)\in\rho$を意味する.
ここで,$\rho$は変数に対応する関数を示す環境であり,\NT{hat variable}と\NT{hat function}からなる二項組の集合として定義される.
以下,\sH{$e$}はHatの式$e$に対する数学的意味を返す意味関数,\sS{$e$}はSchemeの式$e$に対する数学的意味を返す意味関数,\FV{$e$}は式$e$に含まれる自由変数の集合とする.

\NT{hat variable}の意味は次のように定義される.
\begin{eqnarray}
  \sH{$v$}=
  \begin{cases}
  \sH{($f$)} & \llap{if $((v,\exists f)\in\rho)$} \\
  \mbox{wrong ``undefined variable''} \\
  & \llap{otherwise}
  \end{cases}
\end{eqnarray}

\NT{hat function}の意味は以下のように定義される.
\begin{multline}
  \sH{($f$)}= \\
  \begin{cases}
    \lambda x.x(\lambda v_1.\sH{(\hat($v_2\ \cdots\ v_n$) $c$)}) \\
    & \llap{if $(f=\C{\hat($v_1\ v_2\ \cdots\ v_n$) $c$})\land(n\geq 1)$} \\
%    & \llap{$\land(\exists v\not\in\{v_1, v_2,\ldots, v_n\}\cup\FV{$c$})$} \\
    \sH{($c$)} & \llap{if $(f=\C{\hat( ) $c$})$} \\
    \lambda x.x(\lambda v_1.\sH{(\hat($v_2\ \cdots\ v_n\D v_{n+1}$) $c$)}) \\
    & \llap{if $(f=\C{\hat($v_1\ v_2\ \cdots\ v_n\D v_{n+1}$) $c$})$} \\
    & \llap{$\land(n\geq 1)$} \\
%    & \llap{$\land(\exists v\not\in\{v_1, v_2,\ldots, v_n, v_{n+1}\}\cup\FV{$c$})$} \\
    \lambda x.x(\lambda v_1.\sH{(\hat$\ v_2\ c$)}) \\
    & \llap{if $(f=\C{\hat($v_1\D v_2$) $c$})$} \\
%    & \llap{$\land(\exists v\not\in\{v_1, v_2\}\cup\FV{$c$})$} \\
    \lambda v.\sH{($c$)}v & \llap{if $(f=\C{\hat\ $v\ c$})$}
  \end{cases}
\end{multline}

\NT{hat call}の意味は以下のように定義される.
\begin{multline}
  \sH{($c$)}= \\
  \begin{cases}
    \lambda x.x(\sS{$L$}e_1e_2\cdots e_n) \\
    & \llap{if $(c=\C{$L\ e_1\ e_2\ \cdots\ e_n$})$} \\
    \sH{(($f$) $e_1\ e_2\ \cdots e_n$)} \\
    & \llap{if $(c=\C{$v\ e_1\  e_2\ \cdots\ e_n$})$} \\
    & \llap{$\land(n\geq 0)\land((v,\exists f)\in\rho)$} \\
    \sH{(($f$) $e_1\ e_2\ \cdots e_{n-1}$)}(\lambda x.x e_n) \\
    & \llap{if $(c=\C{($f$) $e_1\  e_2\ \cdots\ e_n$})\land(n\geq 1)$} \\
    \sH{($f$)} & \llap{if $(c=\C{($f$)})$} \\
%    \lambda x.x \\
%    & \llap{if $(c=\C{( ) $e_1\  e_2\ \cdots\ e_n$})\land(n\geq 1)$} \\
    \sH{($c_1$\hat($v$) $v\ e_1\ e_2\ \cdots e_n$)} \\
    & \llap{if $(c=\C{($c_1$) $e_1\  e_2\ \cdots\ e_n$})$} \\
    & \llap{$\land (n\geq 1) \land v\not\in$ \FV{($c$)}} \\
    \sH{($c_1$)} & \llap{if $(c=\C{($c_1$)})$} \\
    \lambda x.\sH{($\kappa_1\ e_1\ e_2\ \cdots\ e_n$)}x_1 \\
    & \llap{if $(c=\C{(\FC{$\kappa_1$}{$x_1$}) $e_1\ e_2\ \cdots\ e_n$})$} \\
    \sH{($e_1\ e_2\ \cdots\ e_n$\D($f$))} \\
    % \lambda x.\sH{($e_1\ e_2\ \cdots\ e_n$)}\C{(\FC{($f$)}{$x$})} \\
    & \llap{if $(c=\C{$e_1\ e_2\ \cdots\ e_n\ f$})\land(n\geq 1)$} \\
    \lambda x.\sH{($e_1\ e_2\ \cdots\ e_n$)}\C{(\FC{$\kappa$}{$x$})} \\
    & \llap{if $(c=\C{$e_1\ e_2\ \cdots\ e_n\D\kappa$})\land(n\geq 1)$} \\
  \end{cases}
\end{multline}

\end{document}
